%%\documentclass[fullscreen=true, handout]{beamer}
\documentclass[fullscreen=true]{beamer}
\usetheme[secheader]{Boadilla}
%%\usetheme{Boadilla}
%%\usetheme[height=7mm]{Rochester} 
%%\useinnertheme{rounded}
 \usecolortheme{seahorse}
%% \usecolortheme{whale}
%% \usecolortheme{rose}
%%\setbeamertemplate{navigation symbols}{\insertframenavigationsymbol}
%% or
\beamertemplatenavigationsymbolsempty
%% \setbeamertemplate[info line]{footline} %%Не работает
%% {\quad\strut\insertsection
%% \hfill\insertframenumber/\inserttotalframenumber\strut\quad} 
\usepackage[utf8]{inputenc}
\usepackage[T1]{fontenc} %%apt-get install cm-super
%%\usepackage[english, russian]{babel}
\usepackage[english]{babel}
\emergencystretch=35pt %%Чтобы нe было строчек, вылезших на поля
\usepackage{ragged2e} %% to use \justifying
\usepackage{amsmath}
\usepackage{booktabs}
\usepackage{multicol}
\usepackage{listings}
%% \lstset{language=C++,
%%   backgroundcolor=\color{Black},
%%    basicstyle=\color{White}\tiny\ttfamily,
%%    keywordstyle=\color{Orange},
%%    identifierstyle=\color{Cyan},
%%    stringstyle=\color{Red}, 
%%    commentstyle=\color{Green},  
%%    breaklines=true,
%%    breakatwhitespace=true,
%%    tabsize=10,
%%    showstringspaces=false%%
%% }
%%or better
\lstset{breakatwhitespace,
language=C++,
columns=fullflexible,
keepspaces,
breaklines,
tabsize=2,
backgroundcolor=\color{white},
basicstyle=\color{black}\small\ttfamily,
keywordstyle=\color{orange},
identifierstyle=\color{blue},
stringstyle=\color{red}, 
commentstyle=\color{green},  
showstringspaces=false,
extendedchars=true}
\newcommand{\transhi}{40} %% Переменная - хорошая прозрачность
\setbeamercovered{transparent = \transhi} 
\setbeamertemplate{caption}[numbered]
\usepackage{hyperref}
\begin{document}
%% ----------------------------------------------------------------
\title[\textbf{FDTD 1D example with Python}]{\textbf{FDTD 1D example
    with Python}\\{\small Example task for lectures "Computations in Physics" 
\\ ITMO University, spring 2015}
}

\author[k.ladutenko@metalab.ifmo.ru]{\textbf{Konstantin Ladutenko}%
}

%% ----------------------------------------------------------------
\begin{frame}{}
  \titlepage
  \begin{center}
    All the sources are available at\\
    \textbf{\href{https://github.com/kostyfisik/fdtd-1d}{https://github.com/kostyfisik/fdtd-1d}}
  \end{center}
\end{frame}

\begin{frame}{Contents} %%Добавить интриги
\begin{multicols}{2}
  \tableofcontents%%[pausesections]
\end{multicols}
\end{frame}
\section{Step0: Vanish E field}

\begin{frame}
  \begin{block}{Task 0:}
    Simulate a system with magnetic mirror boundary condition (H=0) on
    one side and electric mirror (E= 0) on the other side. The source
    is a Gaussian profile propagating to boundaries and back, source
    located exactly in the center of the simulated domain. The
    successful presentation should provide a sequence of images as a
    time evolution (or animation) for electric and magnetic
    fields. The simulation should finish at the moment where the
    electric field is vanished (all energy is in the magnetic field).
  \end{block}
\end{frame}

\subsection{step0-at-time-1200.png}
\begin{frame}
  \begin{figure}
    \includegraphics[width=0.75\textwidth]{fig/step0-at-time-1200.png}%
    \caption{Source produces two Gaussian pulses}
  \end{figure}
\end{frame}


\subsection{step0-at-time-2400.png}
\begin{frame}
  \begin{figure}
    \includegraphics[width=0.75\textwidth]{fig/step0-at-time-2400.png}%
    \caption{Before reflection}
  \end{figure}
\end{frame}

\subsection{step0-at-time-4200.png}
\begin{frame}
  \begin{figure}
    \includegraphics[width=0.75\textwidth]{fig/step0-at-time-4200.png}%
    \caption{After reflection}
  \end{figure}
\end{frame}

\subsection{step0-at-time-5998.png}
\begin{frame}
  \begin{figure}
    \includegraphics[width=0.75\textwidth]{fig/step0-at-time-5998.png}%
    \caption{Last step just before vanishing E field}
  \end{figure}
\end{frame}

\subsection{step0-at-time-5999.png}
\begin{frame}
  \begin{figure}
    \includegraphics[width=0.75\textwidth]{fig/step0-at-time-5999.png}%
    \caption{E field vanished. Noise corresponds to numerical accuracy. }
  \end{figure}
\end{frame}

\subsection{step0-at-time-6000.png}
\begin{frame}
  \begin{figure}
    \includegraphics[width=0.75\textwidth]{fig/step0-at-time-6000.png}%
    \caption{The next time step}
  \end{figure}
\end{frame}

\section{Mur ABC and CPML}
\begin{frame}
  \begin{block}{Step 1, 2, and 3:}
    Provide absorbing (Mur ABC) and PML (simplified CPML) boundary
    condition. Compare ABC with  5,10, and 20 cell PML.
  \end{block}
  \begin{block}{Result:}
    For free space Mur ABC is almost perfect. For host media with
    $\epsilon = 5$ the performance is in between 5 and 10 cell CPML,
    20 cell CPML is the best.
  \end{block}
\end{frame}
\subsection{step1-at-time-1680-esp1.png}
\begin{frame}
  \begin{figure}
    \includegraphics[width=0.75\textwidth]{fig/step1-at-time-1680-esp1.png}%
    \caption{Just after reflection of Gaussian pulses using Mur
      boundary condition. Almost perfect for free space.}
  \end{figure}
\end{frame}

\subsection{step1-at-time-3232.png}
\begin{frame}
  \begin{figure}
    \includegraphics[width=0.75\textwidth]{fig/step1-at-time-3232.png}%
    \caption{For the host media with $\epsilon=5$ there is some reflection.}
  \end{figure}
\end{frame}


\subsection{step2-at-time-3232-pml-width-10-eps1.png}
\begin{frame}
  \begin{figure}
    \includegraphics[width=0.75\textwidth]{fig/step2-at-time-3232-pml-width-10-eps1.png}%
    \caption{For free space CPML performs very well too.}
  \end{figure}
\end{frame}

\subsection{step2-at-time-3232-pml-width-10.png}
\begin{frame}
  \begin{figure}
    \includegraphics[width=0.75\textwidth]{fig/step2-at-time-3232-pml-width-10.png}%
    \caption{For $epsilon=5$ 10 cells CPML has one order of magnitude less reflection.}
  \end{figure}
\end{frame}

\subsection{step2-at-time-3232-pml-width-20.png}
\begin{frame}
  \begin{figure}
    \includegraphics[width=0.75\textwidth]{fig/step2-at-time-3232-pml-width-20.png}%
    \caption{20 cells CPML reduces reflection for three orders of
      magnitude compared to 10 cell.}
  \end{figure}
\end{frame}

\subsection{step2-at-time-3232-pml-width-5.png}
\begin{frame}
  \begin{figure}
    \includegraphics[width=0.75\textwidth]{fig/step2-at-time-3232-pml-width-5.png}%
    \caption{5 cells PML performs rather poor.}
  \end{figure}
\end{frame}

\section{Fresnel equations}
\begin{frame}
  \begin{block}{Task 4:}
    Compare against Fresnel equations, find the
    limits of FDTD applicability.
  \end{block}
  \begin{block}{Transmission equation:}
    %$T = 1 -\left| \right$
    \begin{center}
      $T_{Fresnel} = 1-\left|\cfrac{n_1-n_2}{n_1+n_2}\right|^2$
    \end{center}
    \href{http://en.wikipedia.org/wiki/Fresnel_equations}{http://en.wikipedia.org/wiki/Fresnel\_equations}

  \end{block}
\end{frame}


\begin{frame}
  \textbf{Results:}
  \begin{block}{
    $n_1 = 2.828427, n_2 = 1.414214$}
    $T_{Fresnel} = 0.888889$\\
    FDTD ratio = 0.888761 Error = 0.014358\%\\
  \end{block}

  \begin{block}{   $n_1 = 1.732051, n_2 = 2.236068$}
    $T_{Fresnel} = 0.983867$\\
    FDTD ratio = 0.983915 Error =
    0.004879\%\\
  \end{block}
  \begin{block}{$n_1 = 1.000000, n_2 = 3.000000$}
    $T_{Fresnel} = 0.750000$\\
    FDTD ratio = 0.750060 Error = 0.008024\%
  \end{block}
\end{frame}

\section{Dielectric slab}

\begin{frame}
  \begin{block}{Task 5 and 6:}
    Compare against single dielectric slab (e.g
    http://www.ece.rutgers.edu/~orfanidi/ewa/ch05.pdf), you should
    provide simulation of reflection-less cases of a quarter-wavelength
    and half-wavelength slab width cases.
  \end{block}
\end{frame}

\subsection{step5-at-time-2880.png}
\begin{frame}
  \begin{figure}
    \includegraphics[width=0.75\textwidth]{fig/step5-at-time-2880.png}%
    \caption{\textbf{Half-wavelength slab}: switching periodic source
      on. First plane: green curve - wave in free space, blue curve -
      with a slab. Other two are the difference of these curves. Last plane:
      left part of the difference, it is the reflection only. During
      switching on there is some reflection.}
  \end{figure}
\end{frame}

\subsection{step5-at-time-4032.png}
\begin{frame}
  \begin{figure}
    \includegraphics[width=0.75\textwidth]{fig/step5-at-time-4032.png}%
    \caption{\textbf{Half-wavelength slab}: in steady state the
      reflection is reduced 100 times more!}
  \end{figure}
\end{frame}

\subsection{step6-at-time-5070.png}
\begin{frame}
  \begin{figure}
    \includegraphics[width=0.75\textwidth]{fig/step6-at-time-5070.png}%
    \caption{\textbf{Quarter-wavelength slab}: switching on.}
  \end{figure}
\end{frame}

\subsection{step6-at-time-6825-abc.png}
\begin{frame}
  \begin{figure}
    \includegraphics[width=0.75\textwidth]{fig/step6-at-time-6825-abc.png}%
    \caption{\textbf{Quarter-wavelength slab}: steady state}
  \end{figure}
\end{frame}

\subsection{step6-at-time-6825-pml.png}
\begin{frame}
  \begin{figure}
    \includegraphics[width=0.75\textwidth]{fig/step6-at-time-6825-pml.png}%
    \caption{\textbf{Quarter-wavelength slab}: steady state using CPML
    boundary condition instead of Mur ABC. The reflection was reduced
    two times just due to change of computational method.}
  \end{figure}
\end{frame}
\end{document}
